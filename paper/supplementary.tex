\documentclass[letterpaper]{article}
\usepackage{natbib,alifeconf}  %% The order is important
\usepackage{url,hyperref}
\usepackage{amsmath,amssymb}
\usepackage{booktabs}
\usepackage{xcolor}
\usepackage{subcaption}
\usepackage{multirow}

\title{Supplementary Material:\\
Emergent Spatial Coordination from Negative Selection Alone}

\author{Anonymous}

\begin{document}

\maketitle
\IfFileExists{generated/pr26_followups.tex}{%
  % Auto-generated by scripts/render_pr26_followups_tex.py
% source: data/post_hoc/pr26_followups
\newcommand{\PrTwentySixManifestCommit}{63fa9b2e289fb605254545cd48d2e610a5bcf9f2}
\newcommand{\PrTwentySixManifestDoi}{10.5281/zenodo.18713158}
\newcommand{\PrTwentySixFilteredSurvivalPct}{63.55}
\newcommand{\PrTwentySixNoFilterSurvivalPct}{100.00}
\newcommand{\PrTwentySixPhaseTwoKendallTauMedian}{N/A}
\newcommand{\PrTwentySixPhaseTwoTeMedian}{0.0719}
\newcommand{\PrTwentySixPhaseTwoTeNullMedian}{0.0040}
\newcommand{\PrTwentySixPhaseTwoTeExcessMedian}{0.0581}
\newcommand{\PrTwentySixDominantPhenotype}{low\_signal}
\newcommand{\PrTwentySixDominantPhenotypeCount}{44}
\newcommand{\PrTwentySixSynchronousPhaseCount}{3}

}{%
  % Default fallback macros for PR #26 follow-up metrics.
\newcommand{\PrTwentySixManifestCommit}{unknown}
\newcommand{\PrTwentySixManifestDoi}{N/A}
\newcommand{\PrTwentySixFilteredSurvivalPct}{N/A}
\newcommand{\PrTwentySixNoFilterSurvivalPct}{N/A}
\newcommand{\PrTwentySixPhaseTwoKendallTauMedian}{N/A}
\newcommand{\PrTwentySixPhaseTwoTeMedian}{N/A}
\newcommand{\PrTwentySixPhaseTwoTeNullMedian}{N/A}
\newcommand{\PrTwentySixPhaseTwoTeExcessMedian}{N/A}
\newcommand{\PrTwentySixDominantPhenotype}{unknown}
\newcommand{\PrTwentySixDominantPhenotypeCount}{N/A}
\newcommand{\PrTwentySixSynchronousPhaseCount}{N/A}

}

%% =========================================================================
%% A. DENSITY SWEEP ROBUSTNESS
%% =========================================================================
\section{Density Sweep Robustness}
\label{app:density_sweep}

The main experiments use a fixed density of 7.5\% (30 agents on a
$20 \times 20$ grid). To assess robustness across density levels, we
evaluated both Phase~1 and Phase~2 across 12 density conditions: 3 grid
sizes ($15 \times 15$, $20 \times 20$, $30 \times 30$) $\times$ 4 agent
counts (15, 30, 60, 90), yielding densities from 0.017 to 0.400.
Each condition was evaluated with 600 rules (100 rules $\times$ 6 seed
batches), totaling 14{,}400 rule evaluations, all using the Miller-Madow
bias-corrected MI estimator.

\begin{table*}[htbp]
\centering
\caption{Density sweep results across 12 conditions.
$\mathrm{MI}_{\mathrm{excess}}$ values are
Miller-Madow corrected, bits.  Phase~2 achieves nonzero median
$\mathrm{MI}_{\mathrm{excess}}$ in 8 of 12 conditions, including all
conditions with $\geq 60$ agents, while Phase~1 remains at zero across
all densities tested.}
\label{tab:density_sweep}
\begin{tabular}{rllcccc}
\toprule
Density & Grid & Agents & \multicolumn{2}{c}{Median MI\textsubscript{excess} (bits)} & \multicolumn{2}{c}{Survival (\%)} \\
\cmidrule(lr){4-5} \cmidrule(lr){6-7}
 & & & P1 & P2 & P1 & P2 \\
\midrule
0.017 & $30 \times 30$ & 15 & 0.000 & 0.000 & 86.7 & 86.7 \\
0.033 & $30 \times 30$ & 30 & 0.000 & 0.000 & 83.3 & 85.8 \\
0.037 & $20 \times 20$ & 15 & 0.000 & 0.000 & 80.8 & 81.8 \\
0.067 & $15 \times 15$ & 15 & 0.000 & 0.000 & 68.7 & 76.0 \\
0.067 & $30 \times 30$ & 60 & 0.000 & 0.199 & 70.5 & 76.3 \\
0.075 & $20 \times 20$ & 30 & 0.000 & 0.114 & 70.2 & 72.5 \\
0.100 & $30 \times 30$ & 90 & 0.000 & 0.246 & 67.0 & 74.2 \\
0.133 & $15 \times 15$ & 30 & 0.000 & 0.130 & 62.2 & 64.8 \\
0.150 & $20 \times 20$ & 60 & 0.000 & 0.191 & 63.0 & 69.7 \\
0.225 & $20 \times 20$ & 90 & 0.002 & 0.176 & 64.0 & 76.7 \\
0.267 & $15 \times 15$ & 60 & 0.000 & 0.148 & 63.5 & 72.7 \\
0.400 & $15 \times 15$ & 90 & 0.000 & 0.107 & 68.0 & 83.7 \\
\bottomrule
\end{tabular}
\end{table*}

Table~\ref{tab:density_sweep} and Figures~\ref{fig:mi_density}
and~\ref{fig:surv_density} present the results. Phase~2 achieves nonzero
median $\mathrm{MI}_{\mathrm{excess}}$ in 8 of 12 conditions, including all
conditions with $\geq 60$ agents, peaking near $d = 0.100$
($\mathrm{MI}_{\mathrm{excess}} = 0.246$ bits at $30 \times 30$, 90 agents)
before declining at higher densities. Phase~1 remains at zero median
$\mathrm{MI}_{\mathrm{excess}}$ across all 12 conditions, confirming that the
Phase~2 advantage is not an artifact of the specific grid configuration used
in the main experiments.

Phase~2 also consistently achieves higher survival rates than Phase~1, with
the gap widening at higher densities (e.g., 76.7\% vs.\ 64.0\% at $d =
0.225$; 83.7\% vs.\ 68.0\% at $d = 0.400$).

\begin{figure}[htbp]
\centering
\includegraphics[width=0.85\linewidth]{figures/figA1_mi_vs_density.pdf}
\caption{Median neighbor MI vs.\ agent density for Phase~1 and Phase~2.
Phase~2 MI peaks at medium densities and declines at higher densities, while
Phase~1 remains at zero throughout.}
\label{fig:mi_density}
\end{figure}

\begin{figure}[htbp]
\centering
\includegraphics[width=0.85\linewidth]{figures/figA2_survival_vs_density.pdf}
\caption{Survival rate vs.\ agent density. Phase~2 consistently achieves
higher survival than Phase~1, with the gap widening at higher densities.}
\label{fig:surv_density}
\end{figure}

%% =========================================================================
%% B. MULTI-SEED ROBUSTNESS
%% =========================================================================
\section{Multi-Seed Robustness}
\label{app:multi_seed}

The main experiments evaluate each rule table with a single simulation seed.
To assess whether $\mathrm{MI}_{\mathrm{excess}}$ levels are robust
properties of the rule table rather than seed-specific accidents, we selected
the top~50 rules by $\mathrm{MI}_{\mathrm{excess}}$ from each rule-based
condition (Phase~2, Phase~1, and Control) and re-evaluated each across
20 independent simulation seeds.

\begin{table*}[htbp]
\centering
\caption{Multi-seed robustness of top-50 rules per condition (20 seeds each).
Phase~2's elevated MI is a robust rule property; Phase~1 and Control's
low/zero MI are equally stable across seeds.}
\label{tab:multi_seed}
\begin{tabular}{lccc}
\toprule
Metric & Phase~2 & Phase~1 & Control \\
\midrule
Rules with median $\mathrm{MI}_{\mathrm{excess}} > 0$
  & 41/50 (82\%) & 18/50 (36\%) & 22/50 (44\%) \\
Mean $P(\mathrm{MI}_{\mathrm{excess}} > 0)$ across seeds
  & 0.733 & 0.396 & 0.446 \\
Overall survival rate
  & 76.2\% & 90.7\% & 78.7\% \\
\bottomrule
\end{tabular}
\end{table*}

Table~\ref{tab:multi_seed} shows that 82\% of top Phase~2 rules maintain
positive median $\mathrm{MI}_{\mathrm{excess}}$ across seeds, and on average
73.3\% of seeds per rule produce positive excess MI.  In contrast, only 36\%
of top Phase~1 rules and 44\% of top Control rules maintain positive median
$\mathrm{MI}_{\mathrm{excess}}$, with mean positive fractions of 0.396 and
0.446 respectively.  This confirms that MI differences across conditions
reflect genuine rule properties rather than seed-specific accidents: Phase~2's
elevated MI is robust, while Phase~1 and Control's lower MI levels are equally
stable across seeds.

%% =========================================================================
%% C. MORAN'S I (RELOCATED FROM TABLE 1)
%% =========================================================================
\section{Moran's $I$ by Condition}
\label{app:morans_i}

Table~1 in the main text reports the same-state adjacency fraction as the
primary categorical spatial statistic.  For completeness, we report
Moran's~$I$ here, noting that it treats categorical states as numeric
(computing deviations from an arithmetic mean) and is therefore
inappropriate as a primary indicator for nominal data.

\begin{table}[htbp]
\centering
\caption{Median Moran's $I$ by condition (final-step snapshot,
5{,}000 rules per condition).  Moran's $I$ treats states as numeric
and is a secondary indicator; see Table~1 for the categorical
adjacency fraction.}
\label{tab:morans_i}
\begin{tabular}{lc}
\toprule
Condition & Median Moran's $I$ \\
\midrule
Random Walk & $-0.030$ \\
Control     & 0.124 \\
Phase 1     & $-0.011$ \\
Phase 2     & $-0.020$ \\
\bottomrule
\end{tabular}
\end{table}

%% =========================================================================
%% D. HALT WINDOW SENSITIVITY
%% =========================================================================
\section{Halt Window Sensitivity}
\label{app:halt_window}

The main experiments use a 10-step halt window.  To assess sensitivity,
we evaluated the top-50 Phase~2 rules across halt windows of
\{5, 10, 20\} steps.

\begin{table}[htbp]
\centering
\caption{Halt window sensitivity for top-50 Phase~2 rules.
Results are qualitatively unchanged across the tested range.}
\label{tab:halt_window}
\begin{tabular}{ccc}
\toprule
Halt Window & Survival Rate & Median MI\textsubscript{excess} \\
\midrule
5  & 78.0\% & 0.486 \\
10 & 78.0\% & 0.486 \\
20 & 78.0\% & 0.486 \\
\bottomrule
\end{tabular}
\end{table}

Table~\ref{tab:halt_window} confirms that the halt-window parameter has
no impact on the qualitative findings for these top-performing rules:
survival rates (78.0\%) and median MI\textsubscript{excess} (0.486~bits)
are identical across all three tested windows.  This indicates that the
top-50 rules either survive to completion or halt well within the first
5~steps, with no rules in the intermediate regime.

%% =========================================================================
%% E. SURVIVAL RATES WITH CIs
%% =========================================================================
\section{Survival Rates with Confidence Intervals}
\label{app:survival_ci}

\begin{table}[htbp]
\centering
\caption{Survival rates with Wilson score 95\% confidence intervals
(5{,}000 rules per condition).}
\label{tab:survival_ci}
\resizebox{\linewidth}{!}{\begin{tabular}{lccc}
\toprule
Condition & Survived / Total & Rate & 95\% CI \\
\midrule
Random Walk & 5000/5000 & 100.0\% & [99.9, 100.0]\% \\
Control     & 2227/5000 & 44.5\%  & [43.1, 45.9]\% \\
Phase 1     & 3571/5000 & 71.4\%  & [70.1, 72.7]\% \\
Phase 2     & 3735/5000 & 74.7\%  & [73.5, 75.9]\% \\
\bottomrule
\end{tabular}}
\end{table}

%% =========================================================================
%% F. RANDOM WALK DENSITY SWEEP
%% =========================================================================
\section{Random Walk MI\textsubscript{excess} Across Densities}
\label{app:rw_density}

To confirm that the random walk's $\mathrm{MI}_{\mathrm{excess}}$ remains
near zero regardless of agent density, we extended the density sweep to
include the Random Walk condition.  Across all 12 density conditions
(density range 0.017--0.400), the random walk produces
$\mathrm{MI}_{\mathrm{excess}} \approx 0$ (median $\leq 0.06$~bits),
confirming that its elevated raw MI is entirely attributable to
pair-count bias at all tested densities.

%% =========================================================================
%% G. ALTERNATIVE NULL MODELS
%% =========================================================================
\section{Alternative Null Models}
\label{app:alt_nulls}

In addition to the state-shuffle null used throughout the main text, we
evaluated two alternative null models to assess the robustness of the MI
calibration:

\begin{itemize}
  \item \textbf{Block shuffle}: States are shuffled within spatial blocks
        ($4 \times 4$), preserving local autocorrelation structure while
        destroying inter-block correlations.
  \item \textbf{Fixed-marginal}: Synthetic snapshots are generated with
        identical marginal state distributions but independent spatial
        placement (each position draws independently from the observed
        state frequencies).
\end{itemize}

\begin{table}[htbp]
\centering
\caption{Alternative null model comparison for top-50 Phase~2 rules
(mean MI across 200 null samples per rule).  All three null models
produce substantially lower MI than the observed values, confirming
that Phase~2's elevated MI reflects genuine spatial coordination.}
\label{tab:alt_nulls}
\begin{tabular}{lc}
\toprule
Null Model & Mean Null MI (bits) \\
\midrule
State shuffle (main text) & 0.264 \\
Block shuffle ($4 \times 4$) & 0.899 \\
Fixed-marginal & 0.250 \\
\bottomrule
\end{tabular}
\end{table}

The block-shuffle null produces substantially higher MI (0.899~bits) than
the state-shuffle null (0.264~bits), as expected since it preserves
within-block correlations.  The fixed-marginal null (0.250~bits) is
comparable to the state-shuffle.  In all cases, mean observed MI for the
top-50 Phase~2 rules (1.646~bits) substantially exceeds the null values,
confirming genuine spatial coordination.

%% =========================================================================
%% H. SPATIAL SCRAMBLING
%% =========================================================================
\section{Spatial Scrambling Control}
\label{app:spatial_scrambling}

To confirm that Phase~2's elevated MI depends on agents' specific
positions rather than their state distribution alone, we performed
spatial scrambling: for each rule's final snapshot, we randomly
reassigned occupied positions among agents while keeping their states
fixed ($N = 200$ scrambles per rule).

For top-50 Phase~2 rules, the mean observed MI is 1.646~bits while the
mean scrambled MI drops to 0.270~bits---comparable to the shuffle null
baseline (0.264~bits).  This confirms that the observed MI arises from
genuine local spatial coordination (agents with correlated states being
\emph{near} each other) rather than from the state distribution itself.

%% =========================================================================
%% I. TRANSFER ENTROPY
%% =========================================================================
\section{Transfer Entropy}
\label{app:transfer_entropy}

Mutual information measures symmetric statistical dependence between
neighboring states.  To assess \emph{directional} information flow, we
computed transfer entropy (TE) from neighbor states to agent next-states:
\begin{equation}
  \mathrm{TE} = I(S_j^t ; S_i^{t+1} \mid S_i^t)
\end{equation}
where $S_i^t$ is agent $i$'s state at time $t$ and $S_j^t$ is a
neighboring agent's state.  This measures how much knowing a neighbor's
current state reduces uncertainty about the focal agent's next state,
beyond what the agent's own current state provides.

Miller-Madow bias correction is applied.  For the top-50 rules in each
condition, median TE values are: Phase~2 = 0.072~bits, Phase~1 =
0.003~bits, Control = 0.113~bits.  Phase~2 shows substantially elevated
TE compared to Phase~1, confirming directional information flow from
neighbors to agents.  The Control condition's higher TE reflects its
inclusion of a step-clock dimension that creates temporal state
dependence without genuine spatial coordination (recall that Control
$\mathrm{MI}_{\mathrm{excess}} \approx 0$).

%% =========================================================================
%% J. CAPACITY-MATCHED CONTROLS
%% =========================================================================
\section{Capacity-Matched Controls}
\label{app:capacity_matched}

To further isolate the role of observation \emph{content} from table
\emph{capacity}, we evaluated two additional control conditions:

\begin{itemize}
  \item \textbf{Capacity-matched Phase~1}: 100-entry tables where
        indices are aliased so that all dominant-state values for the
        same (self-state, neighbor-count) pair map to the same action.
        This provides Phase-2-sized tables with only Phase-1-level
        observation content.
  \item \textbf{Random-encoding Phase~2}: 100-entry tables with the
        same alphabet size as Phase~2, but the mapping from neighborhood
        configuration to observation index is randomly permuted.  This
        tests whether the \emph{structure} of the encoding matters
        beyond alphabet size.
\end{itemize}

The capacity-matched Phase~1 control produces median
$\mathrm{MI}_{\mathrm{excess}} = 0.000$ (survival 71.4\%), matching
standard Phase~1 ($\mathrm{MI}_{\mathrm{excess}} = 0.000$) and well below
standard Phase~2 ($\mathrm{MI}_{\mathrm{excess}} = 0.096$).  This
confirms that table capacity alone does not explain the Phase~2
advantage---Phase-1-level observations remain ineffective even with
100-entry tables.  The random-encoding Phase~2 control produces median
$\mathrm{MI}_{\mathrm{excess}} = 0.106$ (survival 75.1\%), comparable to
standard Phase~2, which is expected because the observation encoding
(self-state, neighbor count, dominant state) is identical; only the
table-entry ordering differs, which is irrelevant for randomly generated
tables.

%% =========================================================================
%% K. PSEUDOCODE
%% =========================================================================
\section{Algorithmic Pseudocode}
\label{app:pseudocode}

\paragraph{Simulation loop} (\texttt{world.py:step()}).
At each of the 200 time steps:
\begin{enumerate}
  \item Generate a random permutation of agent indices.
  \item For each agent in order:
    \begin{enumerate}
      \item Observe local neighborhood (von~Neumann, 4 cells).
      \item Compute observation vector $(s, n, d)$ or $(s, n)$ depending on
            phase, where $s$ = own state, $n$ = occupied neighbor count,
            $d$ = dominant neighbor state.
      \item Look up action in shared rule table: $a = T[\mathrm{index}(s, n, d)]$.
      \item Execute action: move (if target cell empty), change state, or no-op.
    \end{enumerate}
\end{enumerate}

\paragraph{Filter checks} (\texttt{filters.py}).
After each step, check:
(1) \emph{Halt}: positions and states unchanged for 10 consecutive steps $\to$ terminate.
(2) \emph{State uniform}: all agents share the same state $\to$ terminate.

\paragraph{MI computation} (\texttt{metrics.py}).
For the final snapshot:
\begin{enumerate}
  \item Enumerate all occupied neighbor pairs (right/down on torus, deduped).
  \item Compute joint and marginal state distributions from pairs.
  \item $\hat{I} = \sum p(s_i, s_j) \log_2 \frac{p(s_i, s_j)}{p(s_i)\,p(s_j)}$.
  \item Apply Miller-Madow correction:
        $\hat{I}_{\mathrm{MM}} = \hat{I} - \frac{K_{\mathrm{joint}} - K_X - K_Y + 1}{2n\ln 2}$.
  \item Clamp to $\geq 0$.
\end{enumerate}

\paragraph{Shuffle null} (\texttt{metrics.py}).
Repeat 200 times: permute states among fixed occupied positions, compute
$\hat{I}_{\mathrm{MM}}$, average.
$\mathrm{MI}_{\mathrm{excess}} = \max(\hat{I}_{\mathrm{MM}} - \overline{I}_{\mathrm{shuffle}}, 0)$.

%% =========================================================================
%% L. CROSS-CONDITION METRIC PROFILES
%% =========================================================================
\section{Cross-Condition Metric Profiles}
\label{app:metric_profiles}

Beyond mutual information, the simulation records five additional metric
families (seven individual metrics) at every time step.
Table~\ref{tab:metric_profiles} reports final-step values for surviving
rules across all four conditions.  All pairwise comparisons
(Mann-Whitney $U$, Holm-Bonferroni corrected) are significant at
$p < 0.001$ for every metric.

\begin{table*}[htbp]
\centering
\caption{Cross-condition metric profiles (surviving rules, final-step
values).  Median [Q1, Q3] reported.  All pairwise Mann-Whitney $U$
tests (Holm-Bonferroni corrected) are significant at $p < 0.001$;
effect sizes are reported in the text as signed Cliff's $\delta$
(positive means first-listed group $>$ second-listed group).}
\label{tab:metric_profiles}
\begin{tabular}{lcccc}
\toprule
Metric & Random Walk & Control & Phase~1 & Phase~2 \\
\midrule
Compression ratio
  & 0.178 [0.173, 0.180] & 0.163 [0.158, 0.168] & 0.160 [0.153, 0.168] & 0.160 [0.153, 0.168] \\
Action entropy (mean)
  & 3.141 [3.139, 3.143] & 2.725 [2.555, 2.853] & 0.621 [0.337, 0.958] & 0.778 [0.445, 1.249] \\
Action entropy (var.)
  & 0.000 [0.000, 0.000] & 0.020 [0.008, 0.041] & 0.152 [0.071, 0.234] & 0.190 [0.112, 0.278] \\
Cluster count
  & 29 [28, 30] & 29 [28, 30] & 26 [22, 29] & 27 [24, 29] \\
Quasi-period.\ peaks
  & 4 [3, 5] & 15 [9, 19] & 0 [0, 6] & 2 [0, 7] \\
Phase trans.\ $\max \Delta$
  & 0.266 [0.233, 0.309] & 0.691 [0.572, 0.818] & 0.211 [0.161, 0.327] & 0.260 [0.173, 0.361] \\
State entropy
  & 1.941 [1.897, 1.969] & 1.157 [0.948, 1.446] & 1.295 [0.922, 1.555] & 1.438 [1.091, 1.693] \\
\bottomrule
\end{tabular}
\end{table*}

\paragraph{Role differentiation.}
The variance of per-agent action entropy (action\_entropy\_variance)
captures the degree to which agents specialize into distinct behavioral
roles.  High variance indicates that some agents repeatedly select the
same action while others explore diverse actions---a signature of
emergent role differentiation.  Phase~2 exhibits the highest action
entropy variance (median 0.190), followed by Phase~1 (0.152), while
Control (0.020) and Random Walk ($< 0.001$) show minimal differentiation.
The Phase~1 vs.\ Phase~2 difference is significant (Cliff's
$\delta = -0.185$, $p < 10^{-42}$), confirming that richer
observations support greater role specialization.

\paragraph{Temporal signatures.}
Quasi-periodicity peak count and phase-transition $\max \Delta$ capture
temporal dynamics beyond the MI time-series snapshots in the main text.
Control shows strikingly high quasi-periodicity (median 15 peaks) and
phase-transition $\max \Delta$ (median 0.691), far exceeding Phase~1
(0 peaks, 0.211) and Phase~2 (2 peaks, 0.260).  This reflects
Control's step-clock dimension, which drives periodic state cycling
without genuine spatial coordination ($\mathrm{MI}_{\mathrm{excess}}
\approx 0$).  Phase~1 and Phase~2 show low quasi-periodicity,
consistent with their spatially structured but temporally stable
dynamics.  The Phase~1 vs.\ Phase~2 difference in $\max \Delta$ is
significant (Cliff's $\delta = -0.119$, $p < 10^{-18}$), suggesting
that Phase~2's richer observations produce slightly more dynamic
temporal trajectories.

%% =========================================================================
%% M. CASCADED FILTER ANALYSIS
%% =========================================================================
\section{Cascaded Filter Analysis}
\label{app:cascaded_filters}

The main experiments use only weak (viability) filters: halt detection
and state uniformity.  The codebase also implements medium-strength
filters---short-period detection (period $\leq 2$, checked over 8
snapshots) and low-activity detection (unique-action ratio $< 0.2$ over
5 steps)---which target dynamically trivial but non-halted simulations.

To assess how filter stringency affects the survivor pool, we re-ran
all 5{,}000 rules per condition with both weak and medium filters
enabled (same deterministic seeds, ensuring direct comparability).
Table~\ref{tab:cascade_survival} reports the cascade survival counts.

\begin{table}[htbp]
\centering
\caption{Cascade survival table: weak-only vs.\ weak+medium filters
(5{,}000 rules per condition, same seeds).  Medium filters further
refine the survivor pool while the observation-richness ordering
(Control $<$ P1 $<$ P2) persists.}
\label{tab:cascade_survival}
\resizebox{\linewidth}{!}{\begin{tabular}{lccc}
\toprule
Condition & Weak Only & Weak+Medium & $\Delta$ \\
\midrule
Phase~1 & 3{,}571 (71.4\%) & 2{,}812 (56.2\%) & $-759$ \\
Phase~2 & 3{,}735 (74.7\%) & 2{,}906 (58.1\%) & $-829$ \\
Control & 2{,}226 (44.5\%) & 1{,}915 (38.3\%) & $-311$ \\
\bottomrule
\end{tabular}}
\end{table}

Medium filters remove an additional 15--17\% of rules in Phase~1 and
Phase~2, and 6\% in Control (which already has lower weak-filter
survival).  Crucially, the observation-richness ordering persists:
Phase~2 retains the highest survival rate (58.1\%) among medium-filter
survivors, followed by Phase~1 (56.2\%) and Control (38.3\%).
Furthermore, the median $\mathrm{MI}_{\mathrm{excess}}$ among Phase~2
medium-filter survivors is 0.153~bits, confirming that the MI advantage
is not an artifact of lax filtering.  Phase~1 and Control
medium-filter survivors have median $\mathrm{MI}_{\mathrm{excess}} = 0$.

%% =========================================================================
%% N. PR26 FOLLOW-UP REPRODUCIBILITY SUMMARY
%% =========================================================================
\section{PR26 Follow-Up Reproducibility Summary}
\label{app:pr26_followups}

The post-merge PR26 follow-up bundle is tracked via manifest commit
\texttt{\PrTwentySixManifestCommit} (Zenodo DOI:
\texttt{\PrTwentySixManifestDoi}). The generated summaries report:
filtered survival \PrTwentySixFilteredSurvivalPct\% vs.\ no-filter
\PrTwentySixNoFilterSurvivalPct\%, phase-2 Kendall median
\PrTwentySixPhaseTwoKendallTauMedian, TE median
\PrTwentySixPhaseTwoTeMedian, TE-null median
\PrTwentySixPhaseTwoTeNullMedian, and TE-excess median
\PrTwentySixPhaseTwoTeExcessMedian.

The dominant high-MI phenotype is \texttt{\PrTwentySixDominantPhenotype}
(count \PrTwentySixDominantPhenotypeCount), and synchronous ablation
reports \PrTwentySixSynchronousPhaseCount\ phase-pair summaries.

This archival bundle was generated with the full follow-up configuration
(\texttt{n\_rules=5000}, \texttt{steps=200}) used for manuscript-scale
post-merge PR26 reproducibility.
The phase-2 ranking-stability Kendall median is reported as
\PrTwentySixPhaseTwoKendallTauMedian\ because the compared seed batches
produce no shared surviving rule IDs (\texttt{n\_rules=0} overlap); this value
is intentionally rendered as N/A (non-identifiable correlation), not as a
numeric estimate.

%% =========================================================================
%% O. SHUFFLE-NULL CONVERGENCE
%% =========================================================================
\section{Shuffle-Null Convergence}
\label{app:shuffle_convergence}

To verify that $N=200$ shuffles is sufficient for a stable noise-floor estimate,
we evaluated the mean shuffle-null MI ($\overline{\mathrm{MI}}_\mathrm{null}$)
for the top-50 Phase~2 survivors across $N \in \{10, 25, 50, 100, 200, 500\}$
shuffles per rule. Figure~\ref{fig:shuffle_convergence} shows that the estimate
stabilises by $N \approx 50$ and changes by less than 0.003~bits between $N=100$
and $N=500$, confirming that $N=200$ provides an ample noise-floor estimate with
comfortable margin.

\begin{figure}[htbp]
\centering
\includegraphics[width=0.85\linewidth]{figures/figO1_shuffle_null_convergence.pdf}
\caption{Shuffle-null MI as a function of shuffle count $N$ for the top-50
Phase~2 survivors. Solid line: mean across rules; shaded band: $\pm 1$ SD.
Dashed vertical line marks the default $N=200$ used throughout.}
\label{fig:shuffle_convergence}
\end{figure}

%% =========================================================================
%% P. MUTUAL INFORMATION AS A FUNCTION OF PAIR COUNT
%% =========================================================================
\section{Mutual Information as a Function of Pair Count}
\label{app:mi_vs_n_pairs}

A potential confound is that Phase~2 rules produce more occupied adjacent pairs
($n_\mathrm{pairs}$), which could mechanically increase MI estimates even under
independence. To assess robustness, we computed median $\Delta\mathrm{MI}$ within
$n_\mathrm{pairs}$ bins $\{1\text{--}3,\;4\text{--}6,\;7\text{--}12,\;\geq 13\}$
across all four conditions (Figure~\ref{fig:mi_vs_n}). Phase~2 shows higher median
$\Delta\mathrm{MI}$ than Control and Phase~1 in every bin; it also surpasses Random
Walk in the 4--6, 7--12, and $\geq\!13$ bins (where Phase~2 rules predominantly
reside), confirming that the advantage is not an artifact of pair-count differences.

\begin{figure}[htbp]
\centering
\includegraphics[width=0.85\linewidth]{figures/figP1_mi_vs_n_pairs.pdf}
\caption{Median $\Delta\mathrm{MI}$ (with 95\% bootstrap CI) stratified by
$n_\mathrm{pairs}$ bin for all four conditions. Phase~2's advantage persists
within pair-count strata, ruling out pair-count as a confounding variable.}
\label{fig:mi_vs_n}
\end{figure}

%% =========================================================================
%% Q. POPULATION-LEVEL DELTA-MI(t) TRAJECTORIES
%% =========================================================================
\section{Population-Level $\Delta\mathrm{MI}(t)$ Trajectories}
\label{app:population_delta_mi}

To assess whether the final-snapshot MI advantage of Phase~2 reflects
sustained temporal dynamics or an artefact of measuring coordination at
a single fixed step, we computed the per-step calibrated
$\Delta\mathrm{MI}(t)$ across a random sample of 500 surviving rules per
condition at timestep intervals $t \in \{0, 10, 20, \ldots, 190, 199\}$
(21 checkpoints total). At each checkpoint, the shuffle null is computed
freshly from the recorded snapshot at that step ($N=10$ shuffles;
sufficient for tracking the sign and shape of the population median).
Early-terminated rules are censored at their last recorded step
(MI at last recorded step is used; no imputation).
Figure~\ref{fig:population_delta_mi} shows the median $\Delta\mathrm{MI}(t)$
and interquartile range (IQR) across the 500 sampled rules per condition.

\begin{figure}[htbp]
\centering
\includegraphics[width=0.85\linewidth]{figures/figQ1_population_delta_mi_timeseries.pdf}
\caption{Population-level median $\Delta\mathrm{MI}(t)$ $\pm$ IQR for
500 randomly sampled surviving rules per condition. Solid lines: median
across rules at each timestep; shaded bands: 25th--75th percentile.
Phase~2 (orange) remains above zero throughout all 200 steps (100\% of
timesteps with positive median); Phase~1 (blue) and Control (green) track
at or below zero throughout (0\% of timesteps with positive median).
Censoring policy: early-terminated rules use MI at their last recorded step.}
\label{fig:population_delta_mi}
\end{figure}

Phase~2's median $\Delta\mathrm{MI}(t)$ is positive at 100\% of the 21
timestep checkpoints; Phase~1 and Control achieve positive median
$\Delta\mathrm{MI}(t)$ at 0\% of checkpoints.  This demonstrates that
the Phase~2 advantage is not an artifact of final-snapshot sampling: the
population-wide calibrated coordination is positive and sustained
throughout the simulation.

\paragraph{Archetype frequency by $\Delta\mathrm{MI}$ quartile.}
To characterise what kinds of coordination rules produce across the full
Phase~2 survivor distribution ($n=3{,}735$), we applied the phenotype
taxonomy (\texttt{scripts/phenotype\_taxonomy.py}) to all surviving
Phase~2 rules stratified by $\Delta\mathrm{MI}$ quartile.
Empirical medians: $\Delta\mathrm{MI} = 0.142$~bits, state entropy
$H_{\mathrm{state}} = 1.438$~bits (predictability fraction
$0.142/1.438 \approx 9.9\%$).  Results are shown in
Table~\ref{tab:archetype_quartiles}.

\begin{table}[htbp]
\centering
\caption{Archetype counts per $\Delta\mathrm{MI}$ quartile for all Phase~2
survivors ($n=3{,}735$). Q1 = 0--25th percentile
($\Delta\mathrm{MI} \leq 0.056$~bits); Q2 = 25--50th ($0.056$--$0.142$~bits);
Q3 = 50--75th ($0.142$--$0.437$~bits); Q4 = 75--100th ($>0.437$~bits).
``Low-signal'' is the residual category ($\Delta\mathrm{MI} < 0.05$,
$H<0.60$, or predictability $< 0.40$).}
\label{tab:archetype_quartiles}
\begin{tabular}{lcccc}
\toprule
Archetype & Q1 & Q2 & Q3 & Q4 \\
\midrule
Polarized cluster  & 0 & 0 & 0 & 0 \\
Frozen patch       & 0 & 40 & 4 & 3 \\
Mixed turbulent    & 0 & 5 & 37 & 58 \\
Low signal         & 934 & 889 & 892 & 873 \\
\bottomrule
\end{tabular}
\end{table}

The low-signal category dominates all quartiles, confirming that the majority
of Phase~2 survivors produce modest coordination.  Frozen-patch rules
(static correlated structures) peak in Q2 (median-range ΔMI);
mixed-turbulent rules (dynamic state propagation) are most prevalent in
Q3--Q4 (upper half).  Critically, the top-$\Delta\mathrm{MI}$ quartile (Q4)
contains no polarized-cluster rules, consistent with the anti-correlated
boundary interpretation: high-$\Delta\mathrm{MI}$ rules exhibit near-zero
same-state adjacency fraction, indicating anti-correlation rather than
same-state clustering.

%% =========================================================================
%% R. DISTRIBUTION-WIDE MULTI-SEED ROBUSTNESS
%% =========================================================================
\section{Distribution-Wide Multi-Seed Robustness}
\label{app:random_multi_seed}

Supplementary~\S{}B evaluates multi-seed robustness for the \emph{top-50}
Phase~2 rules by $\Delta\mathrm{MI}$.  Here we extend the analysis to
\emph{200 randomly selected surviving rules} per condition (uniform random
sample, not selected by $\Delta\mathrm{MI}$), re-evaluating each rule
across 10 independent initial seeds.  To quantify sampling uncertainty,
we repeat this procedure with $B=5$ independent random samples (RNG
seeds 0--4) and report bootstrap confidence intervals across resamples.
Control is excluded because rule seeds are not available in the
experiment log for that condition.

\begin{table}[htbp]
\centering
\caption{Distribution-wide multi-seed robustness ($B=5$ resamples of 200
random survivors, 10 seeds per rule). Fraction positive median: fraction
of rules whose median $\Delta\mathrm{MI}>0$ across seeds.
Mean $P(\Delta\mathrm{MI}>0)$: mean fraction of seeds with $\Delta\mathrm{MI}>0$.
95\% CI is the percentile interval across $B=5$ resamples.}
\label{tab:random_multi_seed}
\resizebox{\linewidth}{!}{\begin{tabular}{lcc}
\toprule
Condition & Frac.\ pos.\ med.\ [95\%~CI] & $\overline{P}(\Delta\mathrm{MI}{>}0)$ [95\%~CI] \\
\midrule
Phase~1 & 0.265 [0.230, 0.310] & 0.326 [0.310, 0.351] \\
Phase~2 & 0.709 [0.680, 0.730] & 0.646 [0.627, 0.669] \\
\bottomrule
\end{tabular}}
\end{table}

The Phase~2 fraction with positive median $\Delta\mathrm{MI}$ (70.9\%) is
substantially higher than Phase~1 (26.5\%), with non-overlapping 95\%
bootstrap CIs.  This confirms that seed robustness is a property of the
full Phase~2 survivor distribution, not only the top performers: a random
draw of 200 Phase~2 survivors exhibits the same qualitative ordering as
the top-50 results in \S{}B (Phase 2: 84\%, Phase 1: 36\% among top-50).

%% =========================================================================
%% S. SYNCHRONOUS UPDATE ABLATION
%% =========================================================================
\section{Synchronous Update Ablation}
\label{app:synchronous_ablation}

To assess whether the Phase~2 coordination advantage depends on sequential
update order, we re-ran all 5{,}000 rules per condition under synchronous
update semantics (all agents act simultaneously using the previous step's
state). Table~\ref{tab:synchronous} reports the median $\Delta\mathrm{MI}$,
Cliff's $\delta$, corrected $p$-value, and survival rate for each condition
under both update modes.

\begin{table}[htbp]
\centering
\caption{Synchronous vs.\ sequential update ablation.
Cliff's $\delta$ (sequential vs.\ synchronous); $p_{\rm corr}$ is
Holm-Bonferroni corrected. Phase~2 retains positive median $\Delta\mathrm{MI}$
under synchronous updates (0.013 bits vs.\ 0.096 bits sequential);
Phase~1 and Control remain at zero.}
\label{tab:synchronous}
\begin{tabular}{lccc}
\toprule
Condition & Seq.\ $\Delta\mathrm{MI}$ & Sync.\ $\Delta\mathrm{MI}$ & $p_{\rm corr}$ \\
\midrule
Phase~1  & 0.000 & 0.000 & 0.797 \\
Phase~2  & 0.096 & 0.013 & $1.6\times 10^{-9}$ \\
Control  & 0.000 & 0.000 & --- \\
\bottomrule
\end{tabular}
\end{table}

The Phase~1 and Control conditions show no significant difference between
update modes (Phase~1: Cliff's $\delta=0.002$, $p_{\rm corr}=0.797$).
Phase~2's median $\Delta\mathrm{MI}$ is reduced approximately 7-fold
under synchronous updates (0.013 vs.\ 0.096~bits), but remains positive,
confirming that the directional result (P2~$>$ P1 $=$ Control) is preserved.
The magnitude reduction is consistent with synchronous update eliminating
the implicit temporal ordering that allows state-propagation rules to
propagate information across multiple agents in a single step.

%% =========================================================================
%% T. TERMINATION MODE BREAKDOWN
%% =========================================================================
\section{Termination Mode Breakdown}
\label{app:termination_breakdown}

Table~\ref{tab:termination} breaks down how each of the 5{,}000 rules per
condition terminates.  Halt detection fires when all agents' positions and
states are unchanged for $W=10$ consecutive steps; state-uniformity fires
when all 30 agents share the same internal state.  For the Control condition,
the termination mode is not separately logged; the 2{,}773 non-surviving
rules are reported as ``not recorded.''

\begin{table}[htbp]
\centering
\caption{Termination mode counts per condition ($n=5{,}000$ per condition).
State-unif.\ = state-uniformity termination; N/R = not separately recorded
(Control only).}
\label{tab:termination}
\resizebox{\linewidth}{!}{\begin{tabular}{lrrrr}
\toprule
Condition & Survived & Halt & State-unif. & Total \\
\midrule
Phase~1   & 3{,}571 (71.4\%) & 931 (18.6\%) & 498 (10.0\%) & 5{,}000 \\
Phase~2   & 3{,}735 (74.7\%) & 916 (18.3\%) & 349 (7.0\%) & 5{,}000 \\
Control   & 2{,}227 (44.5\%) & N/R & N/R & 5{,}000 \\
\bottomrule
\end{tabular}}
\end{table}

Control is disproportionately terminated relative to the rule-based conditions
(55.5\% non-survival vs.\ 28.6\% for Phase~1 and 25.3\% for Phase~2).
This is consistent with the Control step-clock producing globally synchronised
state oscillations that rapidly drive agents to uniform state when the
clock cycle aligns across all agents.

\paragraph{Full-distribution $\Delta\mathrm{MI}$ (survived + terminated).}
To confirm that termination does not create a selection artefact, we compute
$\Delta\mathrm{MI}$ at each rule's last recorded step for \emph{all} 5{,}000
rules per condition (censoring policy: MI at last recorded step, no
imputation).  Table~\ref{tab:full_dist_delta_mi} reports the full-distribution
results.

\begin{table}[htbp]
\centering
\caption{Full-distribution $\Delta\mathrm{MI}$ (all 5{,}000 rules per
condition; MI at last recorded step for terminated rules).
The condition ordering Phase~2 $>$ Phase~1 $\approx$ Control is preserved
even when terminated rules are included.}
\label{tab:full_dist_delta_mi}
\begin{tabular}{lcc}
\toprule
Condition & Median $\Delta\mathrm{MI}$ & Fraction $\Delta\mathrm{MI}>0$ \\
\midrule
Phase~1  & 0.000 & 28.6\% \\
Phase~2  & 0.096 & 58.0\% \\
Control  & 0.000 & 14.4\% \\
\bottomrule
\end{tabular}
\end{table}

The Phase~2 median $\Delta\mathrm{MI}$ remains 0.096~bits in the full
distribution (identical to the survivor-only value, because the terminated
rules have near-zero $\Delta\mathrm{MI}$ on average and do not shift the
median substantially).  Phase~1 and Control retain median $\Delta\mathrm{MI}=0$.
State-uniformity filtering removes only zero-MI configurations and cannot
preferentially select high-MI survivors.

%% =========================================================================
%% BIBLIOGRAPHY
%% =========================================================================
\footnotesize
\bibliographystyle{apalike}
\bibliography{references}

\end{document}
