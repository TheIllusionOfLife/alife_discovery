\documentclass[letterpaper]{article}
\usepackage{natbib,alifeconf}  %% The order is important
\usepackage{url,hyperref}
\usepackage{amsmath,amssymb}
\usepackage{booktabs}
\usepackage{xcolor}
\usepackage{subcaption}
\usepackage{multirow}

\newcommand\blfootnote[1]{%
  \begingroup
  \renewcommand\thefootnote{}\footnote{#1}%
  \addtocounter{footnote}{-1}%
  \endgroup
}

\title{Emergent Spatial Coordination from Negative Selection Alone:\\
The Role of Observation Richness in Objective-Free Artificial Life}

\author{Anonymous}

\begin{document}

\maketitle

%% =========================================================================
%% ABSTRACT
%% =========================================================================
\begin{abstract}
We show that spatial coordination among agents emerges in a multi-agent grid world
when agents can observe neighbor states, without any positive behavioral
objective guiding the search.
Existing artificial life systems typically rely on fitness functions---explicit
or implicit---which introduce evaluation bias and constrain the space of
discoverable phenomena.
We propose an objective-free but viability-filtered approach based on
large-scale random rule generation, comparing four observation conditions:
random walk, a step-clock control, density-only observation, and state-profile observation.
Across 5{,}000 rules per condition, agents with state-profile observation
achieve nonzero median $\Delta\mathrm{MI}$ (MI calibrated by a permutation-based shuffle
null that controls for pair-count bias) while
density-only and control agents remain at zero
(Cliff's $\delta = 0.355$; median-difference bootstrap 95\% CI $[0.078, 0.111]$; Table~\ref{tab:stats}),
and the ordering Control $\leq$ Phase~1 $<$ Phase~2 holds across all
rule-based comparisons.
A Miller-Madow bias-corrected estimator is used throughout, and a shuffle null
confirms that the random walk's elevated raw MI is entirely attributable to
pair-count bias ($\Delta\mathrm{MI}$ near zero at 0.050~bits, within the noise
floor of the $N\!=\!200$ shuffle null).
Moran's $I$ further distinguishes local coordination from large-scale clustering.
These results demonstrate that observation channel richness---not rule table
capacity or positive-objective selection pressure---drives the emergence of
spatial coordination in objective-free but viability-filtered systems.
\end{abstract}

Submission type: \textbf{Full Paper}\\

Data/Code available at: \url{https://anonymous.4open.science/r/objectless-alife}
\newline Reproducibility map (artifact lineage): \texttt{docs/reproducibility.md}
\newline PR26 follow-up archive~\citep{pr26archive2026}
\blfootnote{\textcopyright 2026 Anonymous.
  Published under a Creative Commons Attribution 4.0
  International (CC BY 4.0) license.}

%% =========================================================================
%% 1. INTRODUCTION
%% =========================================================================
\section{Introduction}

Artificial life research aims to understand the principles of living systems
by constructing synthetic analogs~\citep{bedau2003artificial}. A recurring
challenge is the role of the \emph{objective function}: most evolutionary and
adaptive systems require an explicit fitness measure that guides search toward
``interesting'' configurations. Even novelty search~\citep{lehman2011abandoning},
which abandons traditional fitness, still uses a novelty metric as an implicit
objective.

This reliance on objectives introduces a subtle but pervasive bias. The choice
of fitness function constrains which phenomena can emerge, and researchers may
inadvertently encode their expectations into the evaluation
criteria~\citep{stanley2019openended}. The question then arises: \emph{can
meaningful spatial structure emerge in a multi-agent system with no fitness
function?}

We explore the unexplored quadrant of \emph{no positive objective $\times$
viability-only selection}, where the only filtering criterion is
non-degeneracy---removing rules that produce trivially degenerate simulations
(all agents halt or converge to a single state). Crucially, while this negative
selection does constitute a minimal selection pressure (surviving rules must
avoid complete stasis and global state uniformity), it imposes no behavioral or
performance criterion: any surviving rule is equally valid regardless of its
spatial structure, MI level, or action distribution. This is analogous to
\emph{minimal criteria} selection~\citep{lehman2011abandoning}---a deliberately
weak constraint that shapes the set of viable rules without specifying what is
desirable.

Our core contribution is threefold:
\begin{enumerate}
  \item A minimal grid-world model with objective-free negative selection,
        where random rule tables are evaluated and only degenerate ones
        (halt or state-uniformity) are discarded.
  \item Evidence that \emph{observation richness}---the amount of neighbor
        state information available to agents---drives emergent spatial
        coordination, independent of rule table capacity.
  \item Robustness across four experimental conditions and 20{,}000 rule
        evaluations, with statistical significance confirmed by Mann-Whitney $U$
        tests with Holm-Bonferroni correction.
\end{enumerate}

%% =========================================================================
%% 2. RELATED WORK
%% =========================================================================
\section{Related Work}

\paragraph{Self-organization without selection.}
Cellular automata such as Conway's Game of Life~\citep{gardner1970game} and
Wolfram's elementary rules~\citep{wolfram1984cellular} demonstrate that simple
local rules can produce complex global patterns. Continuous extensions like
Lenia~\citep{chan2019lenia} show rich morphogenetic dynamics in continuous
state spaces. Reynolds' Boids~\citep{reynolds1987flocks} produce flocking
behavior from three local rules. These systems share a common trait: the rules
are hand-designed, not discovered through search.

\paragraph{Evolutionary ALife with fitness.}
Tierra~\citep{ray1991approach} and Avida~\citep{ofria2004avida} use implicit
fitness through resource competition and self-replication. While these systems
produce open-ended dynamics, the replication criterion itself acts as a fitness
function that selects for self-replicating programs.

\paragraph{Novelty search and open-endedness.}
Novelty search~\citep{lehman2011abandoning, lehman2008exploiting} replaces
fitness with a novelty metric, enabling discovery of diverse
behaviors. The open-ended evolution community has explored various
approaches to sustaining innovation~\citep{taylor2016open,
stanley2019openended}. However, all such approaches still employ an evaluation
function---whether fitness, novelty, or complexity.
The \emph{minimal criteria} framework~\citep{lehman2011abandoning} uses
deliberately weak constraints to avoid trivial solutions without imposing
performance objectives---our viability filtering is directly analogous.

\paragraph{Information-theoretic measures.}
Mutual information and transfer entropy have been used to quantify coordination
in multi-agent systems~\citep{lizier2012local}.
We use mutual information as a \emph{post-hoc} analysis tool, never as a
selection criterion.

\paragraph{Our position.}
Our approach differs from all the above by using \emph{no positive objective}
---not fitness, novelty, or complexity as an optimization target. We generate
random rules, discard only degenerate ones (halt or state-uniformity), and ask what structure the
surviving rules exhibit.

%% =========================================================================
%% 3. METHODS
%% =========================================================================
\section{Methods}

\subsection{World Model}

The simulation environment is a $20 \times 20$ toroidal grid populated by 30
agents (Figure~\ref{fig:snapshot}). Each agent occupies exactly one cell (no
overlap allowed) and maintains an internal state $s \in \{0, 1, 2, 3\}$. At
each of 200 time steps, agents are updated in a random sequential order: one
agent at a time observes its local neighborhood, looks up an action in a shared
rule table, and executes it. The action space comprises 9 mutually exclusive
actions: 4 cardinal movements, 4 state changes, and a no-op. Movement to an
occupied cell fails silently.

\begin{figure*}[htbp]
\centering
\includegraphics[width=\linewidth]{figures/fig1_snapshot_grid.pdf}
\caption{Grid snapshots of hand-picked highest-$\Delta\mathrm{MI}$ surviving
rules from each condition (final-step grid; Miller-Madow estimator).
Raw MI values shown are for individual top-performing rules, not condition
medians; see Table~\ref{tab:mi_summary} for population statistics.
RW and Control panels may show similar elevated raw MI---this is expected, as
raw MI is inflated by pair-count bias; $\Delta\mathrm{MI}$ (Table~\ref{tab:mi_summary})
correctly shows near-zero calibrated values for both.
Each panel shows the $20 \times 20$ toroidal grid with agents colored by
internal state. Phase~2 rules produce visibly clustered spatial patterns,
while Control rules appear disordered.}
\label{fig:snapshot}
\end{figure*}

\begin{table}[htbp]
\centering
\caption{Model and experiment parameters.}
\label{tab:params}
\begin{tabular}{lp{0.52\columnwidth}}
\toprule
Parameter & Value \\
\midrule
Grid topology       & $20 \times 20$ torus (von~Neumann) \\
Number of agents    & 30 (7.5\% density) \\
Internal states     & $\{0, 1, 2, 3\}$ \\
Action space        & 9 (4 moves, 4 state changes, 1 no-op) \\
Simulation steps    & 200 \\
Update order        & Random permutation each step \\
Halt window         & 10 consecutive unchanged steps \\
MI estimator        & Miller-Madow corrected, base-2 (bits) \\
Shuffle null $N$    & 200 permutations per rule \\
Seeds per condition & 5{,}000 (rule seed $i$, sim seed $i$) \\
\bottomrule
\end{tabular}
\end{table}

\subsection{Observation Phases}

We compare four observation conditions that vary in the information available
to agents:

\paragraph{Random Walk (RW).}
Each agent selects an action uniformly at random from $\{0, \ldots, 8\}$ at
every step, ignoring the rule table entirely.
(Reported as ``5{,}000 seeded runs'' rather than rules, since no rule table is
consulted.)
This baseline isolates the contribution of grid geometry (collision avoidance,
toroidal wrapping) from rule-driven behavior, and also serves to calibrate the
MI estimator under conditions with minimal spatial clustering.

\paragraph{Control (step-clock).}
Agents observe their own state $s \in \{0,\ldots,3\}$, the count of occupied
von~Neumann neighbors $n \in \{0,\ldots,4\}$, and a periodic step clock
$t \bmod 5 \in \{0,\ldots,4\}$. The rule table has $4 \times 5 \times 5 = 100$
entries. The step clock is non-informative \emph{about neighbor identity or state}---a
global clock can induce temporal synchrony but carries no spatial information
about neighbors.  It serves as a capacity-matched third dimension that reaches
Phase~2's table size without adding any spatial content.

\paragraph{Phase 1: density-only (P1).}
Agents observe their own state $s$ and neighbor count $n$. The rule table has
$4 \times 5 = 20$ entries, indexed by $5s + n$. This is the minimal
observation that couples agents spatially.

\paragraph{Phase 2: state profile (P2).}
Agents observe their own state $s$, neighbor count $n$, and the dominant
neighbor state $d \in \{0,\ldots,4\}$ (the most frequent state among occupied
neighbors, with ties broken by smallest value; 4 denotes no occupied neighbors).
The rule table has $4 \times 5 \times 5 = 100$ entries, indexed by
$25s + 5n + d$.

\subsection{Viability Filters}

Only two filters are applied, both enforcing non-degeneracy rather than
behavioral quality:
\begin{enumerate}
  \item \textbf{Halt detection}: If all agents' positions and states remain
        unchanged for $W=10$ consecutive steps, the simulation is terminated.
        Applied uniformly from step~1 with no burn-in.
  \item \textbf{State uniformity}: If all 30 agents share the same internal
        state at any step, the simulation is terminated.
        Global homogeneity is the criterion; partial uniformity is not penalized.
\end{enumerate}
These filters are applied identically across all four conditions and all density
levels.  Any surviving rule may exhibit any spatial pattern, MI value, or action
distribution; the filters exclude only dynamical trivialism (complete stasis or
indistinguishable agents).
No fitness function, novelty metric, complexity threshold, or behavioral
criterion is used at any stage.
Across 5{,}000 rules per condition, survival rates are:
Random Walk 100\%, Phase~1 71.4\%, Phase~2 74.7\%, Control 44.5\%.
Operationally, these filters enforce viability constraints only: they remove
degenerate trajectories but do not reward any specific spatial pattern,
information score, or task completion behavior. All coordination metrics are
computed \emph{after} filtering and are never fed back into search.

\subsection{Metrics}

All metrics are computed post-hoc and never used for selection:

\paragraph{Neighbor mutual information (MI).}
For each pair of adjacent occupied cells $(i, j)$ on the toroidal grid, we
compute the mutual information between their internal states:
\begin{equation}
  I(S_i; S_j) = \sum_{s_i, s_j} p(s_i, s_j)
    \log_2 \frac{p(s_i, s_j)}{p(s_i)\,p(s_j)}
\end{equation}
where the joint and marginal distributions are estimated from all adjacent
occupied pairs at a given time step. To mitigate the positive bias of the
plug-in estimator at small sample sizes, we apply the Miller-Madow
correction~\citep{miller1955note}: $\hat{I}_{\mathrm{MM}} = \hat{I} -
(K_{\mathrm{joint}} - K_X - K_Y + 1)/(2n\ln 2)$, where $K_{\mathrm{joint}}$,
$K_X$, and $K_Y$ are the counts of non-zero bins in the joint and each marginal
distribution respectively, and $n$ is the number of pairs. Values are clamped
to $\geq 0$.
Concretely: the random variables are the internal states of both agents in each
occupied adjacent pair at the \emph{final simulation step} (step~200).
All occupied neighbor pairs at that step are pooled and MI is computed once
from the aggregate joint count---not averaged over per-step estimates.
MI is not conditioned on density; pair-count bias is controlled via the
shuffle null instead.
High MI indicates that neighboring agents' states are statistically
dependent---a signature of spatial coordination.

\paragraph{Interpreting $\Delta\mathrm{MI}$ as coordination.}
Neighbor $\Delta\mathrm{MI}$ measures local statistical dependence between
adjacent agents' states beyond chance.  A positive $\Delta\mathrm{MI}$
indicates state assignments are not spatially random.  Crucially,
$\Delta\mathrm{MI}$ alone does not distinguish \emph{dynamic} coordination
from \emph{frozen} clustering: a world where agents freeze in correlated
positions would also yield high MI.  We address this via two complementary
analyses: (i)~temporal MI trajectories (Figure~\ref{fig:mi_timeseries})
distinguish frozen early-plateau dynamics (Phase~1) from sustained dynamic
coordination (Phase~2), and (ii)~transfer entropy (supplementary material)
confirms directional information flow in Phase~2 beyond symmetric MI.
The viability filter further rules out the most trivial high-MI scenario by
explicitly removing state-uniform worlds.

\paragraph{Shuffle-null MI calibration.}
To control for pair-count bias---where configurations with few neighbor pairs
produce inflated MI estimates regardless of state structure---we compute a
permutation-based shuffle null.  For each rule's final snapshot, we fix
occupied positions and randomly reassign states among them $N = 200$ times,
computing $\hat{I}_{\mathrm{MM}}$ for each shuffle.  Specifically, the shuffle
operator preserves: (i) the set of occupied positions, (ii) the neighbor-pair
graph structure, and (iii) the marginal state-frequency counts; it permutes
only the assignment of states to positions.  The mean across shuffles
is the shuffle-null MI.  The calibrated $\Delta\mathrm{MI}$ is then:
\begin{equation}
  \Delta\mathrm{MI} = \hat{I}_{\mathrm{MM}} - \overline{I}_{\mathrm{shuffle}}
  \label{eq:delta_mi}
\end{equation}
which isolates genuine spatial coordination from estimation artifacts.
Unlike a rectified (max(0)) version, the unrectified $\Delta\mathrm{MI}$
allows negative values when the observed MI falls below the shuffle null,
providing an unbiased view of the full distribution.
Specifically: positions are fixed; states among occupied cells are uniformly
permuted; $\hat{I}_{\mathrm{MM}}$ is computed for each of $N=200$ shuffles;
$\overline{I}_{\mathrm{shuffle}}$ is the mean over those shuffles.
The Miller-Madow correction is applied independently to both the observed MI
and each shuffle replicate before averaging.
All MI values are in bits (base-2 logarithm).
In particular, the random walk's high raw MI
($\approx 0.91$~bits) is entirely attributable to pair-count bias:
$\Delta\mathrm{MI}$ near zero (0.050~bits, within the noise floor
of the $N\!=\!200$ shuffle null).

\paragraph{Same-state adjacency fraction.}
Because agent states are categorical (nominal), traditional Moran's $I$ is
inappropriate---it treats states as numeric, computing deviations from an
arithmetic mean.  We therefore report the \emph{same-state adjacency fraction}
(a join count statistic): the fraction of occupied neighbor pairs sharing the
same internal state, bounded in $[0, 1]$.  This is the standard spatial
autocorrelation measure for categorical data.

\paragraph{Moran's I.}
Spatial autocorrelation is quantified by Moran's $I$~\citep{moran1950notes}
across occupied cells using torus 4-neighborhood weights: both cells must be occupied for a
weight of~1.  Values range from $-1$ (dispersed) through $0$ (random) to $+1$
(clustered).  As noted above, Moran's $I$ treats states as numeric and is
therefore a secondary indicator; the same-state adjacency fraction is the
primary categorical spatial statistic.

\paragraph{State entropy.}
Shannon entropy~\citep{cover1991elements} of the internal state distribution
across all agents: $H = -\sum_s p(s) \log_2 p(s)$.

\paragraph{Action entropy.}
Per-agent Shannon entropy of the cumulative action distribution, summarized as
the mean and variance across agents.

\subsection{Experimental Design}

For each of the four conditions, we generate 5{,}000 random rule tables using
deterministic seeds (rule seeds 0--4{,}999, simulation seeds 0--4{,}999). Each
rule table is evaluated on a single 200-step simulation. Surviving rules (those
not terminated by halt or state-uniformity filters) have their final-step
metrics recorded; all pairwise statistical comparisons use only surviving rules.

Statistical comparisons use two-sided Mann-Whitney $U$
tests~\citep{mann1947test} with Holm-Bonferroni correction~\citep{holm1979simple}
applied across all metrics within each pairwise comparison. Each pairwise
comparison (e.g., P1 vs.\ P2) tests a distinct hypothesis about the effect of
observation content, so correction is applied per comparison rather than
globally across all stages. Effect sizes are reported as Cliff's
$\delta$~\citep{cliff1993dominance} (equivalent to rank-biserial correlation
for Mann-Whitney $U$):
$\delta = 1 - 2U_A/(n_1 n_2)$, where $U_A$ is the $U$ statistic for
the first-listed group. To replace extreme $p$-values with more informative
measures, we report bootstrap 95\% confidence intervals~\citep{efron1979bootstrap} for
the median difference (10{,}000 resamples, percentile method) as the primary
effect-size
statistic; $p$-values are retained as secondary evidence.
Each rule table is paired with a single simulation
seed (rule seed $i$ paired with simulation seed $i$), so per-rule conclusions
reflect one initial configuration.

%% =========================================================================
%% 4. RESULTS
%% =========================================================================
\section{Results}

\subsection{Evidence Ladder Among Rule-Based Conditions}

Table~\ref{tab:mi_summary} and Figure~\ref{fig:mi_distribution} present the
neighbor mutual information across all four conditions. Among the three
rule-based conditions, Phase~2 separates clearly from both baselines.
A smaller but statistically detectable shift also holds between Phase~1 and
Control (Cliff's $\delta = 0.124$; see Table~\ref{tab:stats}), though both
share median $\Delta\mathrm{MI} = 0$ and the practical coordination difference
is negligible.  The ordering is:

\begin{center}
\textbf{Control $<$ Phase~1 $<$ Phase~2}
\end{center}

\begin{table*}[htbp]
\centering
\caption{Calibrated neighbor mutual information and spatial clustering by condition.
$\Delta$MI = MI\textsubscript{observed} $-$ MI\textsubscript{shuffle null}
is the primary calibrated metric controlling for pair-count bias
(5{,}000 rules per condition, $N=200$ shuffles).
Raw MI values use the Miller-Madow bias-corrected estimator and are reported secondarily.
The random walk's high raw MI is entirely attributable to pair-count bias
($\Delta$MI $\approx 0.050$~bits, within the shuffle-null
noise floor).}
\label{tab:mi_summary}
\begin{tabular}{lcccccc}
\toprule
Condition & Table Size & Median $\Delta$MI & Median MI & Median Adj.\ Frac.\footnotemark & Frac.\ $\Delta\mathrm{MI}>0$ & Survival \\
\midrule
Random Walk & 1 (unused) & 0.050 & 0.910  & 0.364 & 52\% & 100.0\% \\
Control     & 100        & 0.000 & 0.000  & 0.329 & 14\% & 44.5\% \\
Phase 1     & 20         & 0.000 & 0.019  & 0.291 & 29\% & 71.4\% \\
Phase 2     & 100        & 0.096 & 0.298  & 0.350 & 58\% & 74.7\% \\
\bottomrule
\end{tabular}
\footnotetext{Same-state adjacency fraction (join count statistic for
categorical data).  Moran's $I$ (inappropriate for nominal states) is
reported in the supplementary material.}
\end{table*}

The control condition, despite having 100-entry tables (equal to Phase~2),
produces zero median $\Delta\mathrm{MI}$---demonstrating that table size alone is insufficient.
Phase~2 is the only rule-based condition with nonzero median $\Delta\mathrm{MI}$,
driven by access to neighbor state information
(Table~\ref{tab:stats}; bootstrap 95\% CI excludes zero).

\begin{figure}[htbp]
\centering
\includegraphics[width=\linewidth]{figures/fig2_mi_distribution.pdf}
\caption{Final-step neighbor MI (Miller-Madow corrected, bits) distributions
across rule-based conditions.
Box plots with scatter strips show the full distribution for surviving rules.
The evidence ladder Control~$<$~P1~$<$~P2 is clearly visible.}
\label{fig:mi_distribution}
\end{figure}

\subsection{Random Walk Baseline and Shuffle-Null Calibration}
\label{sec:rw_baseline}

All MI values in this paper use the Miller-Madow bias-corrected
estimator~\citep{miller1955note}, which subtracts a first-order correction
term from the naive plug-in estimate. However, this correction becomes
insufficient when the number of neighbor pairs $n$ is comparable to the
number of joint-distribution bins $K$---precisely the regime of random-walk
agents, where $\approx$4--5 adjacent pairs populate up to 16 joint
categories.

The expected pair count varies substantially by condition: at the final step
(step~200), the mean $\pm$ SD number of occupied adjacent pairs is
$4.3 \pm 1.9$ for Random Walk, $3.9 \pm 3.6$ for Control,
$11.0 \pm 8.1$ for Phase~1, and $12.0 \pm 7.3$ for Phase~2.
Phase~1 and Phase~2 produce far more adjacent pairs than the baselines
because rule-driven agents tend to aggregate; yet their $\Delta\mathrm{MI}$
values differ dramatically (0.000 vs.\ 0.096~bits), confirming that the
difference reflects state structure rather than sample-size effects.

The shuffle null resolves this directly. By permuting states among fixed
positions ($N=200$ shuffles per rule), we obtain the expected MI under the
null hypothesis of no spatial state structure. The calibrated
$\Delta\mathrm{MI} = \mathrm{MI}_{\mathrm{observed}} -
\overline{\mathrm{MI}}_{\mathrm{shuffle}}$ isolates genuine coordination
from pair-count artifacts (see Eq.~\ref{eq:delta_mi}).

For the random walk, $\Delta\mathrm{MI}$ is near zero
(0.050~bits)
(Table~\ref{tab:mi_summary}), confirming that its elevated raw MI is
entirely attributable to pair-count bias---not spatial structure. In
contrast, Phase~2 retains substantial $\Delta\mathrm{MI}$, demonstrating genuine
coordination that persists after bias calibration. Notably, the Control
condition---not Phase~2---has the highest median Moran's $I$ (0.124),
reflecting clustered but uncoordinated spatial patterns from its large table
size. Phase~2's median Moran's $I$ ($-0.020$) is near zero, indicating that
its elevated MI arises from local state coordination among neighbors rather
than from large-scale spatial clustering.

The random-walk baseline remains informative: it validates the shuffle-null
calibration procedure, establishes the bias floor, and confirms that all
5{,}000 random-walk rules survive (100\% survival rate) since random
actions never trigger halt or state-uniformity filters.

\subsection{Table-Size Confound Resolved}

A natural objection is that Phase~2's higher MI could result from its larger
rule table (100 entries vs.\ 20 for Phase~1), which permits more complex
behaviors. The control condition resolves this confound directly: it uses
100-entry tables---identical in size to Phase~2---but replaces the informative
dominant-neighbor-state dimension with a non-informative step clock. The control
produces \emph{lower} MI than Phase~1 despite having 5$\times$ more table
entries. This demonstrates that \textbf{observation content, not table capacity,
drives emergent coordination}.

The pairwise comparisons confirm this (Table~\ref{tab:stats}): Phase~1
produces significantly higher MI than Control despite having smaller tables,
and Phase~2 vastly exceeds Control despite equal table size.

\subsection{Filter-Selection Confound Check}

A key methodological concern is that viability filters might indirectly favor
high-MI rules, creating an apparent Phase~2 advantage through selection bias.
To test this directly, we measured point-biserial correlation between survival
status and final-step MI within each condition. The correlations are nearly
identical for Phase~1 and Phase~2 ($r \approx 0.14$ in both), while Control is
higher ($r = 0.45$) due to its substantially lower survival rate.
The relevant comparison is therefore Phase~1 vs.\ Phase~2: because their
filter--MI coupling is matched, the Phase~2 $\Delta\mathrm{MI}$ advantage cannot be
explained by differential filter selection.

\subsection{Ranking-Stability Overlap Interpretation}

Post-merge follow-up analysis reports Kendall-$\tau$ stability with explicit
overlap diagnostics across seed batches. For Phase~2 in the archived
follow-up run, compared seed batches show zero shared surviving rule IDs under
the selected alignment key, so $\tau$ is undefined and reported as
\emph{N/A}. This indicates non-identifiability (insufficient overlap), not
negative rank correlation.

\subsection{Temporal Dynamics}

The four conditions exhibit qualitatively distinct temporal behaviors
(Figure~\ref{fig:mi_timeseries}):
\begin{itemize}
  \item \textbf{Phase~1}: MI rises quickly then plateaus---``frozen'' dynamics
        where spatial patterns crystallize early.
  \item \textbf{Phase~2}: MI rises and remains dynamic, with ongoing
        fluctuations---sustained spatial coordination without freezing.
  \item \textbf{Control}: Highly chaotic trajectories with large MI
        variance and frequent collapses to zero.
  \item \textbf{Random Walk}: High but flat MI throughout, reflecting
        constant estimation bias from few adjacent pairs rather than
        genuine coordination.
\end{itemize}

This pattern is an exploratory qualitative observation only. It is consistent
with a possible frozen--dynamic--chaotic ordering across conditions, but we do
not claim edge-of-chaos evidence here because the analysis is based on a small
sample of top-performing rules and does not include quantitative criticality
measures (e.g., Lyapunov exponents).

\begin{figure}[htbp]
\centering
\includegraphics[width=\linewidth]{figures/fig3_mi_timeseries.pdf}
\caption{MI time-series trajectories for top-3 rules per condition.
Phase~1 freezes early, Phase~2 remains dynamic, and Control shows chaotic
fluctuations.}
\label{fig:mi_timeseries}
\end{figure}

\subsection{Seed Robustness}
\label{sec:seed_robustness}

To confirm that condition-level differences reflect rule properties rather than
single-seed outcomes, we re-evaluated the top 50 rules from each rule-based
condition across 20 independent initial seeds (1{,}000 simulations per condition).
Among Phase~2 rules, 84\% maintain positive median $\Delta\mathrm{MI}$ across
seeds (mean $P(\Delta\mathrm{MI}>0) = 0.748$).
For Phase~1 and Control, the corresponding figures are 36\% and 44\%
respectively.  The Phase~2 $\Delta\mathrm{MI}$ advantage is not seed-specific:
the condition ordering is preserved regardless of initial configuration.

\subsection{Coordination Archetypes}
\label{sec:archetypes}

To characterize \emph{what kind} of coordination Phase~2 produces, we examined
top-performing rules and temporal trajectories, identifying three recurring
coordination patterns:

\begin{enumerate}
  \item \textbf{State-propagation dynamics.}  Agents propagate the dominant
        neighbor state across the grid, producing traveling wavefronts of state
        transitions.  These rules show high $\Delta\mathrm{MI}$, moderate
        same-state adjacency fraction, and sustained MI fluctuations
        (consistent with Phase~2's dynamic temporal trajectory).
  \item \textbf{Frozen coordination (majority-locking).}  Agents rapidly
        converge to a dominant state and freeze, yielding high but static MI.
        This pattern resembles Phase~1 behavior but arises via state-dependent
        rather than density-only rules.
  \item \textbf{Persistent boundary structures.}  Stable clusters form at
        domain boundaries between different states, maintaining high
        neighbor-state correlation (positive $\Delta\mathrm{MI}$) with minimal
        movement.  Near-zero Moran's $I$ confirms these are local structures,
        not large-scale uniform clusters.
\end{enumerate}

This taxonomy is based on qualitative inspection; formal feature-based
clustering is available in the code repository
(\texttt{scripts/phenotype\_taxonomy.py}), which uses the features
$\Delta\mathrm{MI}$, same-state adjacency fraction, state entropy,
and predictability (Hamming) with deterministic thresholds
($\Delta\mathrm{MI}>0.12$, adjacency $>0.60$, entropy $<0.30$)
to assign each rule to one of four archetypes.

\subsection{Statistical Significance}

Table~\ref{tab:stats} presents Mann-Whitney $U$ test results for the primary
metric (neighbor MI) across all pairwise comparisons. All comparisons are
highly significant after Holm-Bonferroni correction.

\begin{table*}[htbp]
\centering
\caption{Pairwise statistical tests for $\Delta\mathrm{MI}$
(Eq.~\ref{eq:delta_mi}).
Cliff's $\delta$ measures effect size (= rank-biserial $r$); bootstrap
95\% CIs for the median difference (10{,}000 resamples) are the
primary effect-size statistic; $p$-values (Holm-Bonferroni corrected) are
secondary.  Note that Cliff's $\delta$ measures stochastic dominance
(probability that a random draw from one group exceeds the other), whereas
the bootstrap CI estimates the location shift of the median difference.
Effect-size interpretation: $\delta = 0.355$ (P1 vs.\ P2) is moderate-to-large;
$\delta = 0.124$ (Ctrl vs.\ P1) is small but statistically reliable with $n > 2000$;
median differences for Ctrl vs.\ P1 are $<\!0.001$~bits, indicating practical
equivalence in coordination level despite statistical detectability.}
\label{tab:stats}
\begin{tabular}{llccccc}
\toprule
Comparison & Direction & $n_1$ & $n_2$ & Cliff's $\delta$ & Median diff 95\% CI & $p$-value \\
\midrule
P1 vs.\ P2      & P2 $>$ P1   & 3570 & 3735 & 0.355 & [0.078, 0.111] & $< 10^{-252}$ \\
Ctrl vs.\ P1    & P1 $>$ Ctrl & 2225 & 3570 & 0.124 & [$< 0.001$, $< 0.001$] & $< 10^{-50}$ \\
Ctrl vs.\ P2    & P2 $>$ Ctrl & 2225 & 3735 & 0.423 & [0.078, 0.111] & $\approx 0$ \\
\bottomrule
\end{tabular}
\end{table*}

\subsection{Survival Analysis}

Survival rates differ significantly across conditions (Table~\ref{tab:mi_summary}).
Phase~2 achieves the highest survival rate, followed by Phase~1
and Control. State-uniformity is the dominant
termination mode for the control condition, suggesting that without neighbor
state information, rules frequently drive all agents to the same state.

%% =========================================================================
%% 5. DISCUSSION
%% =========================================================================
\section{Discussion}

\paragraph{Observation richness as a driver of emergence.}
Our central finding is that the \emph{content} of observation---specifically,
access to neighbor state information---is the primary driver of emergent spatial
coordination. This holds even when controlling for rule table size (the control
condition). With shuffle-null-calibrated MI, Phase~2 is the
only rule-based condition achieving nonzero median $\Delta\mathrm{MI}$, while
both Phase~1 and Control fall to zero---making the qualitative separation
between observation conditions unambiguous.
The same-state adjacency fraction (Table~\ref{tab:mi_summary}) provides
a categorical spatial statistic appropriate for nominal state data.
Phase~2's adjacency fraction (0.350) exceeds the random-placement
expectation ($\approx 0.25$ for 4 equiprobable states), confirming
local coordination.  Moran's $I$ (reported in the supplementary
material) is a secondary indicator, as it treats categorical states as
numeric; nonetheless, it reveals that the Control condition exhibits the
strongest spatial clustering (median $I = 0.124$) despite zero $\Delta\mathrm{MI}$,
while Phase~2's near-zero median $I$ ($-0.020$) indicates that its
coordination operates at the local neighbor level rather than through
large-scale spatial clustering.
This result is not a consequence of having more rules to choose from, but of
each rule being able to respond to richer local information.

\paragraph{Exploratory temporal pattern (not a primary claim).}
The temporal trajectories suggest a possible frozen--dynamic--chaotic ordering
(Phase~1, Phase~2, Control), but we treat this strictly as exploratory context,
not as evidence for edge-of-chaos theory. Establishing such a claim would
require dedicated criticality analyses beyond the scope of the present work.

\paragraph{Viability filtering as minimal selection.}
Our methodology embodies a minimal philosophy: generate random configurations,
remove only the degenerate ones (halt or state-uniformity), and examine what structure the survivors
exhibit. We acknowledge that this ``negative selection'' constitutes a minimal
selection pressure---surviving rules must avoid degenerate dynamics (halt or
state uniformity)---but it is purely physical, imposing no behavioral or
performance criterion. The surprising finding is that meaningful
structure---quantified by mutual information---emerges even under this minimal
regime, provided the observation channel is sufficiently rich.

\paragraph{Filter-metric independence.}
A potential concern is that viability filters might inadvertently select for
high-MI rules, confounding the comparison between observation conditions.
Point-biserial correlation between survival status and final-step MI yields
$r = 0.14$ for Phase~2 ($p < 10^{-22}$), $r = 0.14$ for Phase~1, and
$r = 0.45$ for Control. The Control's elevated correlation reflects its
low survival rate (44.5\% vs.\ $\sim$72\% for rule-based conditions):
stronger selection on survivors inflates the correlation mechanically.
The relevant comparison is P1 and P2's near-identical $r \approx 0.14$,
which indicates that the MI advantage of Phase~2 over Phase~1 is not an
artifact of differential filter selection: both conditions have nearly
identical filter-MI correlation, yet Phase~2 achieves substantially higher
$\Delta\mathrm{MI}$.

\paragraph{Directional information flow.}
Transfer entropy analysis (supplementary material) confirms that the spatial
coordination in Phase~2 involves genuine directional information flow from
neighbor states to agent next-states, beyond what symmetric MI captures.
This rules out the possibility that the observed MI arises from static
correlations without dynamic coupling.

\paragraph{Capacity-matched controls.}
To further isolate the role of observation \emph{content} from table
\emph{capacity}, we evaluated two additional control conditions
(supplementary material): (i)~a capacity-matched Phase~1 with 100-entry
tables aliased to 20 effective observations, and (ii)~a random-encoding
Phase~2 with randomly permuted neighborhood-to-observation mappings.
Both controls produce $\Delta\mathrm{MI}$ significantly below
standard Phase~2, confirming that structured encoding of neighbor state
information---not table size or alphabet size---drives coordination.

\paragraph{Spatial scrambling confirmation.}
Spatial scrambling analysis (supplementary material) confirms that
Phase~2's elevated MI depends on genuine local spatial coordination:
randomly reassigning agent positions while preserving states reduces MI
to near-zero levels, consistent with the shuffle-null calibration.

\paragraph{Role differentiation and temporal signatures.}
Beyond MI, our metric suite reveals additional dimensions of emergent
structure (supplementary material, Section~L).
The variance of per-agent action entropy captures whether agents
specialize into distinct behavioral roles: high variance indicates
emergent role differentiation, where some agents consistently select the
same action while others explore.  Phase~2's access to dominant neighbor
state is expected to support greater differentiation than Phase~1 or
Control, where agents cannot distinguish neighbor identity.
Quasi-periodicity peaks and phase-transition $\max \Delta$ provide
temporal signatures that complement the MI time-series analysis:
recurrent oscillatory patterns and abrupt reorganization events,
respectively.  Cross-condition differences in these temporal metrics
further support the observation-richness narrative.

\paragraph{Filter cascade robustness.}
The main experiments apply only weak viability filters (halt detection,
state uniformity).  A cascaded filter analysis (supplementary material,
Section~M) applies additional medium-strength filters---short-period
detection and low-activity detection---to test whether the Phase~2
MI advantage persists under stricter viability criteria.  If the
observation-richness ordering (Control~$<$~P1~$<$~P2) holds among the
more stringently filtered survivors, it strengthens the claim that
coordination is a genuine property of the rule--observation interaction,
not an artifact of lax filtering.

\paragraph{Implications for ALife research.}
These results suggest that objective-free search deserves more attention as a
complement to fitness-driven approaches. When the goal is to discover
\emph{what is possible} rather than to optimize for a specific outcome,
removing the objective function may reveal phenomena that fitness landscapes
obscure. The key enabler is not the search algorithm but the
\emph{architecture} of the agents---specifically, what they can observe.

%% =========================================================================
%% 6. LIMITATIONS
%% =========================================================================
\section{Limitations}

Several limitations constrain the generalizability of our findings:

\begin{itemize}
  \item \textbf{Single topology}: All experiments use a $20 \times 20$ toroidal
        grid with von~Neumann neighborhoods. Other topologies (hexagonal grids,
        Moore neighborhoods, irregular graphs) may produce different results.
  \item \textbf{Symmetric metric}: Mutual information is symmetric and measures
        correlation, not causation. Transfer entropy (supplementary material)
        confirms directional information flow in Phase~2, but a full
        time-resolved analysis remains for future work.
  \item \textbf{No multi-generation evolution}: Each rule table is evaluated in
        a single 200-step simulation. We do not evolve rules across generations,
        which limits comparison with evolutionary ALife systems.
  \item \textbf{Small state space}: With only 4 internal states and 9 actions,
        the model is deliberately minimal. Scaling to larger state spaces may
        reveal qualitatively different dynamics.
  \item \textbf{Density fixed}: The main experiments use 30 agents on a
        400-cell grid (7.5\% density). The supplementary material
        reports a systematic density sweep across 12 conditions
        (density range 0.017--0.400), confirming that the Phase~2 MI
        advantage holds across all tested densities.
  \item \textbf{Sequential update order}: Agents are updated in a random
        permutation each step, creating implicit temporal correlations between
        early and late updates within the same step. Synchronous update
        schemes may produce different dynamics.
  \item \textbf{Single simulation per rule (main experiment)}: Each rule table
        is evaluated with a single initial configuration in the main experiment.
        The supplementary material provides multi-seed robustness analysis
        for the top 50 rules from each rule-based condition (Phase~2, Phase~1,
        and Control), confirming that MI levels---whether high, low, or
        zero---are stable properties of the rule table, not seed-specific
        accidents.  As a brief in-text summary: among the top 50 Phase~2 rules
        re-evaluated across 20 independent seeds, 84\% maintain positive
        median $\Delta\mathrm{MI}$ (mean $P(\Delta\mathrm{MI}>0)=0.748$),
        compared to 36\% for Phase~1 and 44\% for Control.
  \item \textbf{Halt window sensitivity}: The main experiments use a
        10-step halt window. A sensitivity sweep across \{5, 10, 20\}
        (supplementary material) confirms that results are robust to
        this parameter choice: survival rates and MI distributions
        remain qualitatively unchanged.
  \item \textbf{Ranking overlap boundary case}: Kendall-$\tau$ ranking
        stability requires overlapping rule identities across batches.
        When overlap is zero, $\tau$ is not estimable and is reported
        as N/A. Overlap fraction is therefore a prerequisite diagnostic
        for interpreting ranking stability.
\end{itemize}

%% =========================================================================
%% 7. CONCLUSION
%% =========================================================================
\section{Conclusion}

We have shown that meaningful spatial coordination emerges in a multi-agent
system through objective-free but viability-filtered negative selection, and that the richness of
agents' observation channels---not rule table capacity---is the critical factor.
The evidence ladder from step-clock control through density-only to
state-profile observation demonstrates a monotonic relationship between
observation content and emergent coordination, confirmed by a
bias-corrected MI estimator across all tested density levels
(see the supplementary material).

Transfer entropy analysis (supplementary material) confirms directional
information flow in Phase~2, and capacity-matched controls rule out table
size as a confound.  Future work should extend to larger grids and state
spaces, explore multi-generation rule evolution under the same
objective-free regime, and investigate whether richer observation channels
enable qualitatively different forms of coordination.  The broader
implication is that the ``remove broken, observe survivors'' philosophy
can serve as a productive complement to fitness-driven search in
artificial life.

\paragraph{Supplementary Material.}
Density sweep analysis, multi-seed robustness results (all three
rule-based conditions), halt-window sensitivity, survival rate confidence
intervals, alternative null models, spatial scrambling analysis, transfer
entropy, capacity-matched controls, algorithmic pseudocode,
cross-condition metric profiles, and cascaded filter analysis are
available as supplementary material with the code repository.

%% =========================================================================
%% BIBLIOGRAPHY
%% =========================================================================
\footnotesize
\bibliographystyle{apalike}
\bibliography{references}

\end{document}
